\section{Fazit}
Der Arbeitstitel dieser Bachelor-Thesis war \dq{}\myTitel\dq{} und hatte als Ziel Testautomatisierung für eine bestehende Shopware Schnittstelle zu analysieren und diese umzusetzen.

Nach Klärung der notwendigen theoretischen Grundlagen 

Die vorhandenen Testszenarien waren von dem Vorgängerdienstleister kaum bis gar nicht dokumentiert. Eine Erweiterung der Schnittstelle war ohne Regressionen gar nicht mehr möglich da alle vorhandenen Funktionen vom Umfang her nicht mehr manuell testbar waren. Die Aktualisierung des Shopsystems glich dem Griff in eine Wundertüte. Der Kunde wusste nachher nie welche Funktionalitäten seiner Schnittstelle noch gegeben waren.

Die detaillierte Ist- und Schwachstellenanalyse haben deutlich gemacht, das in der heutigen Zeit Software mit einem rein manuellem Testingansatz nicht mehr handhabbar sind. Ohne eine ergänzende, idealerweise Vollständige automatisierte Testingstrategie kann eine Software nicht mehr gepflegt und weiterentwickelt werden.

Als Ausblickend betrachtet bieten 