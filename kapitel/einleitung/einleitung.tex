\section{Einleitung}
\subsection{Ausgangssituation und Problemstellung}
Ausgehend von einer entwickelten ERP-Schnittstelle für die Verknüpfung eines Warenwirtschaftssystems und einem Online-Shop die nur eine manuelle Testabdeckung aufweist, werden mit dieser Bachelor-Thesis die Testprozesse analysiert und eine neue Schnittstelle mit konsequenter automatisch Testabdeckung von der best it GmbH \& Co. KG umgesetzt. Die best it GmbH \& Co. KG ist eine der führenden eCommerce Agenturen im deutschsprachigen Raum, die sich auf die Entwicklung und Digitalisierung von Geschäftsmodellen im B2C- und B2B-Handel spezialisiert hat. Die best it GmbH \& Co. KG pflegt eine langjährige Partnerschaft mit der Shopware AG, die zeitgleich auch der Hersteller der gleichnamigen Online-Shop Software Shopware ist. 

Der Kunde hat sich an die best it GmbH \& Co. KG gewandt, um die vorhandene Schnittstelle zu analysieren und ggf. neu entwickeln zu lassen. Da der neu geworbene Kunde schon bereits einen Shopware Online-Shop hat, war die Ist-Analyse ohne größere Einarbeitung durchführbar. Das Resultat ergab, dass die bestehende Schnittstelle des Kunden gravierende Mängel auf weißt. Durch die Art der Schnittstellenumsetzung lassen sich notwendige Aktualisierung des eingesetzten Online-Shop System nur schwer und mit hohem finanziellen Aufwand einspielen. Zusätzlich sind Funktionserweiterungen auch mit hohen monetären Investitionen verbunden. Der vorherige Dienstleister hat die umgesetzten Funktionalitäten der Schnittstelle nur spärlich dokumentiert. Eine detaillierte Testanweisung der Schnittstelle fehlt gänzlich. 

Der Online-Shop Betreiber steht durch den immer größer werdenden Wettbewerb sowie durch eine immer größere Preissensetivität der Käufer unter immensem Kosten- und Leistungsdruck. Die Prozesse für die Logistik und den Versand sowie Stammdatenpflege und Bestandsaktualisieren müssen automatisiert sein, um eine effizientes und somit auch ein Kostengünstiges Wirtschaften zu ermöglichen. Eine wichtige, wenn nicht die wichtigste Voraussetzung dafür, stellt eine funktionierende und verlässliche bidirektionale Schnittstelle zwischen Online-Shop und Warenwirtschaft dar.

Die aktuelle Schnittstelle weißt hohe Wartungskosten auf. Um diese mittel- und langfristig zu reduzieren hat der Kunde die best it GbmH \& Co. KG beauftragt die Schnittstelle vollständig neu zu entwickeln mit dem Fokus auf Wartbar- und Erweiterbarkeit um den sich immer schneller verändernden E-Commerce Markt gerecht zu werden. 

\subsection{Zielsetzung}
Aufbauend auf den theoretischen Grundlagen wird mit dieser Bachelor-Thesis die Zielsetzung verfolgt, die aktuellen Probleme und Schwachstellen der eingesetzten Schnittstelle zu ergründen. Um fundierte Informationen für ein realisierbares Soll-Konzept zu haben soll eine SWOT-Analyse bezüglich dem Thema der Testautomatisierung durchgeführt werden. Dies stellt die Entscheidungsgrundlage da um die neu zu entwickelnde Schnittstelle mit einer Testautomatisierung zu versehen und den anfallenden Mehraufwand gegenüber dem Kunden begründen zu können. 

Die neue Schnittstelle soll, durch die in der SWOT-Anlayse bestätigten Stärken der Testautomatisierung, mit einer hohen Testabdeckung entwickelt werden um die geforderten Anforderungen des Kunden zu erfüllen. Um den nun erreichten Mehrwert für den Kunden zu Verfügung zustellen, ist ein Mutation-Testing Szenario basierend auf den erstellten Quellcode und die daraus resultierenden Tests angedacht. Die dadurch entstehende Metrik liefert dem Kunden einen neutralen Wert wie gut die Testautomatisierung wirklich ist. 

\subsection{Aufbau der Arbeit}
Zuerst werden die notwendigen Theoretischen Grundlagen in Kapitel \ref{theoretische-grundlagen} erläutert. Ein Schwerpunkt dieses Kapitels stellen die verschiedenen Testarten, Clean-Code und Software-Metriken dar.

In dem anschließendem Kapitel befindet sich eine detaillierte Ist- und Schwachstellenanalyse der aktuell eingesetzten Schnittstelle. Darauf aufbauend wurde eine SWOT-Analyse durchgeführt um zu evaluieren welche Verbesserungen die Testautomatisierung für den Kunden ermöglicht. Zeitgleich stellt diese Analyse weitere Fakten für ein fundiertes Soll-Konzept zur Verfügung welches im anschließendem Kapitel erstellt wurde.

Als weiteren Schwerpunkt der Arbeit folgt nun in Kapitel \ref{umsetzung} die Umsetzung der neuen Schnittstelle. Zu erst werden die Strukturellen Voraussetzungen erläutert und die notwendigen Testwerkzeuge ausgewählt. Anschließend geht es direkt in die technische Umsetzung der Schnittstelle wo dort die Kernthemen der Softwarearchitektur und der verschiedenen Testarten Anwendung finden.

Zu Schluss wird ein detailliertes Fazit gezogen ob die umgesetzte Schnittstelle den Anforderungen des Kunden entsprechen. Außerdem findet ein Ausblick in die mögliche Zukunft für den Kunden statt.


%\subsection{Grenzen und Bedingungen der Arbeit}

%\subsection{Aufbau der Arbeit}
